\capitulo{1}{Introducción}

En el ámbito universitario anualmente se generan gran cantidad de documentos académicos que proporcionan una valiosa fuente de información. Especialmente las memorias de los \textbf{Trabajos de Fin de grado (TFG)} que poseen un alto valor y que forman parte fundamental del proceso formativo del estudiante, ya que permiten aplicar los conocimientos adquiridos durante la realización de los estudios de grado. Estas memorias contienen información sobre tendencias tecnológicas actuales, herramientas y investigaciones realizadas sobre distintas áreas de la informática. Sin embargo, resulta muy laborioso ubicar y extraer esta información ya que encuentra almacenada de manera dispersa en distinto repositorios y formatos.

Este trabajo tiene como finalidad el desarrollo de una aplicación para el análisis cualitativo de los textos contenidos en las memorias de TFG del grado en ingeniería Informática de la Universidad de Brugos. Esta herramienta permite el análisis, consulta, facilita la exploración, calculo de tendencias y comparación de trabajos a lo largo del tiempo partiendo de estos contenidos textuales.  El conjunto de datos analizado se encuentra en formato LaTeX y se encuentra disponible en plataformas como GitHub, a estos datos se le aplicaran técnicas de modelado de temes utilizando distintos algoritmos de última generación, entre ellos Latent Dirichlet Allocation (LDA), Top2Vec, BERTopic y FASTopic.
 
La aplicación resultante servirá como ayuda a la comunidad docente, investigadora e estudiantil para la exploración de la información académica. Ademas, se busca poner en practica los conocimiento adquiridos y competencias durante la titulación.


La memoria se estructura de la siguiente manera: en el \textbf{Capítulo 2} se presenta el marco teórico y los fundamentos del modelado de temas; en el \textbf{Capítulo 3} se describe la metodología seguida y el diseño del sistema; en el \textbf{Capítulo 4} se exponen los resultados obtenidos y su análisis comparativo; y finalmente, en el \textbf{Capítulo 5} se detallan las conclusiones y las posibles líneas de trabajo futuro.



Descripción del contenido del trabajo y del estructura de la memoria y del resto de materiales entregados.
